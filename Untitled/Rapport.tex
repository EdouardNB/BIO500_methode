\documentclass[9pt,twocolumn,twoside,]{pnas-new}

% Use the lineno option to display guide line numbers if required.
% Note that the use of elements such as single-column equations
% may affect the guide line number alignment.


\usepackage[T1]{fontenc}
\usepackage[utf8]{inputenc}

% tightlist command for lists without linebreak
\providecommand{\tightlist}{%
  \setlength{\itemsep}{0pt}\setlength{\parskip}{0pt}}




\templatetype{pnasresearcharticle}  % Choose template

\title{Template for preparing your research report submission to PNAS
using RMarkdown}

\author[a]{Pénélope Robert}
\author[a]{Édouard Nadon-Beaumier}
\author[a]{Alexis Matte}
\author[a]{Nadia Tardy}

  \affil[a]{Université de Sherbrooke, Départment de bio, 2500 Boulevard
de l'Université, Sherbrooke, Québec, G1V 0A9}


% Please give the surname of the lead author for the running footer
\leadauthor{}

% Please add here a significance statement to explain the relevance of your work
\significancestatement{}


\authorcontributions{}



\correspondingauthor{\textsuperscript{} }

% Keywords are not mandatory, but authors are strongly encouraged to provide them. If provided, please include two to five keywords, separated by the pipe symbol, e.g:


\begin{abstract}
Please provide an abstract of no more than 250 words in a single
paragraph. Abstracts should explain to the general reader the major
contributions of the article. References in the abstract must be cited
in full within the abstract itself and cited in the text.
\end{abstract}

\dates{This manuscript was compiled on \today}
\doi{\url{www.pnas.org/cgi/doi/10.1073/pnas.XXXXXXXXXX}}

\begin{document}

% Optional adjustment to line up main text (after abstract) of first page with line numbers, when using both lineno and twocolumn options.
% You should only change this length when you've finalised the article contents.
\verticaladjustment{-2pt}



\maketitle
\thispagestyle{firststyle}
\ifthenelse{\boolean{shortarticle}}{\ifthenelse{\boolean{singlecolumn}}{\abscontentformatted}{\abscontent}}{}

% If your first paragraph (i.e. with the \dropcap) contains a list environment (quote, quotation, theorem, definition, enumerate, itemize...), the line after the list may have some extra indentation. If this is the case, add \parshape=0 to the end of the list environment.

\acknow{}

La notion du réseau écologique d'un écosystème est souvent déformé et
confondu avec celle du réseau trophique. Nous avons régulièrement
tendance à imager les interations entre les différents espèces avec une
belle pyramide qui retrace les réseaux trophiques et la chaîne
alimentaire en plaçant le plus important prédateur au sommet du prisme.
En réalité, un réseau écologique inclu toutes les multitidudes
interactions possibles entre les espèces comme le mutualisme,la
compétition, le commensalisme et le parasitisme. Évidement, les réseaux
trophiques ont une grande influence sur le réseaux écologiques, mais ce
dernier ne devrait pas être représenter par une pyramique. Il devrait
plutôt être décrite comme une toile (``web'') avec les espèces clés de
l'écosystème au centre, c'est-à-dire les espèces qui ont le plus grand
nombre de connexions interspécifiques. C'est dans ce contexte que nous
avons voulu comparer le réseau de collaboration entre les étudiants en
écologie et de celles des réseaux écologiques. Afin d'appronfondir notre
réflexion, nous avons voulu vérifié si le nombre de collaborations par
étudiants de BIO500 sont semblables ou bien s'il y a des ``étudiants
clés,'' c'est-à-dire qui possèdent le plus de collaborations différentes
au même titre qu'une espèce clé.

Avant même de se pencher sur nos questions, il a fallu que chacun des
étudiants de BIO500 recensent l'entièreté des étudiants avec qui il ou
elle a collaboré dans un travail d'équipe dans leur parcours académique
en biologie à l'Université de Sherbrooke. Une fois les données
recueillies et traitées, nous avions toutes les informations en main
pour commencer notre analyse. Nous avons en premier lien émis une
hypothèse. Selon nous, le nombre de collaboration par personne devrait
être environ semblable d'un étudiant à l'autre puisque le parcours
académique et les cours sont relativement les mêmes. Nous nous attendons
donc à ce que le nombre de collaborations par étudiant suive une
distribution normale. Si ce n'est pas le cas, nous nous pencherons sur
l'hypothèse inverse des ``étudiants clés.'' L'objectif sera de les
identifiés et de déterminer la cause de leur haut nombre de
collaborateurs.

Nous avons en premier lieu une analyse visuel et graphique et en
deuxième lieu une analyse statistique de notre distribution. Les deux
analyses, comme les traitements de données ci-haut, ont été effectué sur
le logiciel R version 4.0.3. le L'analyse visuel est représenté par un
histogramme, voici 4 résultats possibles ainsi que leur signification:

\emph{Une courbe symétrique et mince}

C'est à quoi nous nous attendions. La variance entre le nombre de
collaboration entre les élèves de BIO500 est faible et très concentré
autour de la moyenne. Les étudiants ont donc eux un parcours académique
semblable.

\emph{Une courbe symétrique et large}

Cette courbe corrobore également notre hypothèse: une distribution
normale. Par contre, la variation entre les élèves est plus grande
qu'attendu. Il a une petite divergence entre les parcours, mais qui sans
débalancer les données.

\emph{Une courbe asymétrique à gauche}

Ce genre de distribution non normal indique que la majorité des élèves
se retrouve malgré tout proche de la moyenne. Cependant, un certain
nombre non négligeable d'élèves ont plus que collaboration et qu'aucun
ou très peu d'élèves ont peu collaboré. Cette courbe nous indique qu'il
a plusieurs ``étudiants clés'' plus ou moins important.

\emph{Une courbe symétrique avec des données aberrantes vers la droite}

On observe une distribution normale, mais avec ce qui semble être des
données aberrantes. Cela représente précisément l'autre hypothèse
``d'étudiants clé'' peu nombreux et très important. Leur divergence
distinctive pourrait signifier une grande différence de leur parcours.

L'analyse statistique est effectuée par le test de Shapiro-Wilk dans
lequel les hypothèses sont les suivantes:

\begin{itemize}
\tightlist
\item
  Hypothèse nulle (H0): les données ont une distribution normale
\item
  Hypothèse alternative (H1): les données n'ont pas une distribution
  normale
\end{itemize}

La commande shapiro.test() est facile d'utilisation et sa sortie nous
informe sur la valeur de W et le p-value. C'est seulement le p-value qui
nous intéresse. Si p-value est plus petit que 0.05, on rejette
l'hypothèse nulle: les données sont non normales. Si p-value est plus
grand que 0,05, on ne rejette pas l'hypothèse nulle: les données ont
donc une distribution normale.



% Bibliography
% \bibliography{pnas-sample}

\end{document}
